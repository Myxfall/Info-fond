\documentclass[a4paper,10pt]{article}
\usepackage[utf8]{inputenc}
\usepackage[francais]{babel}
\usepackage{indentfirst}
\usepackage{listings}
\usepackage{graphicx}
\usepackage{blindtext}
\usepackage{enumitem}
\usepackage{hyperref}
\usepackage{amsmath}
\usepackage[top=2.5cm,bottom=2.5cm,left=2.5cm,right=2.5cm]{geometry}
\pagestyle{headings}
\title{Projet : INFO-F-302 Informatique Fondamentale.}
\author{George Rusu et Maximilien Romain}
\date{\today}
\begin{document}
\maketitle
\tableofcontents
\newpage
\section{Introduction}
Le premier objectif de ce projet est de modeliser divers problemes en problemes de satisfaction de contraintes (CSP). Le second objectif est d’implementer un programme resolvant ces problemes en utilisant ChocoSolver.

\textbf{link} :  http://www-master.ufr-info-p6.jussieu.fr/2005/IMG/pdf/rp3.pdf

\section{Question 1}
L'ensemble des cases du jeux $V$ où $ \#V = n^2$ \\
\textbf {Variables de décision} $ X = \{x_{i,j} | \forall i,j (1 \leq i \leq n).(1 \leq j \leq n)\}$ , $n^2$ variables de décision \\
\textbf{Domaines : }
$ D = (vide, fous, cavalier, tour) $

\textbf{Contraintes : }\\
Contrainte{ Xij si Di=F alors {pour nimporte quel i=j -> xij}}
Contrainte{ Xij si Di=T alors {pour un i,j -> tout les j xij et tout les i de xij}}
Contrainte{ Xij si Di=c alors {pour un i,j -> tout les j de xij + tout les i de xin, tout les i de xij + tout les j de xnj}}

$ \{ x_{i,j} \wedge tour \rightarrow x_{i,j} (\forall algo crois) \wedge vide \} $

$\forall x_{i,j} \in D_4 : \exists l=n : (\forall x_{l,j} \in D_1, l \neq j) \wedge (\forall x_{i,l} \in D_1, l \neq i)$

$ c_1 = (\{x_1, x_2\},(b_i, b_j)| b_i = T, b_j = vide \lor b_i = vide, b_j = T) $

\textbf{Contraintes tour}\\
$ c_{T_{col},j} = ((x_{1,j}, x_{2,j},\ldots, x_{n,j}), \{(b_1, b_2,\ldots, b_n) | b_i = T, b_j = V, \forall j \ne i\})$ pour les colonnes \\
$ c_{T_{ligne},j} = ((x_{i,1}, x_{i,2},\ldots, x_{i,n}), \{(b_1, b_2,\ldots, b_n) | b_i = T, b_j = V, \forall j \ne i\})$ pour les lignes

\textbf{Contraintes fou}\\
$ c_{F,2*n-2} = ((x_{1,n-1}, x_{2,n}), \{(b_1, b_2)| b_1 = F, b_2 = V \lor b_1 = V, b_2 = F\}) \land ((x_{1,2}, x_{2,1}), \{(b_1, b_2)| b_1 = F, b_2 = V \lor b_1 = V, b_2 = F\}) $ \\
\vdots \\
$ c_{F,n+1} = ((x_{1,2}, x_{2,3}, \ldots, x_{n-1,n}), \{(b_1, b_2, \ldots, b_{n-1})|b_i = F, b_j = V, \forall j \ne i\}) \land ((x_{1,n-1},x_{2,n-2}, \ldots, x_{n-1,1}), \{(b_1, b_2, \ldots, b_{n-1})|b_i = F, b_j = V, \forall j \ne i\}) $ \\
$ c_{F,n} = ((x_{1,1}, x_{2,2},\ldots, x_{n,n}), \{(b_1, b_2, \ldots, b_n)|b_i = F, b_j = V, \forall j \ne i\}) \land ((x_{1,n}, x_{2,n-1}, \ldots, x_{n, 1}), \{(b_1, b_2, \ldots, b_n)|b_i = F, b_j = V, \forall j \ne i\})$ \\
$ c_{F,n-1} = ((x_{2,1}, x_{3,2},\ldots, x_{n, n-1}), \{(b_1, b_2, \ldots, b_{n-1})|b_i = F, b_j = V, \forall j \ne i\}) \land ((x_{2,n}, x_{3,n-1}, \ldots, x_{n,2}), \{(b_1, b_2, \ldots, b_{n-1})|b_i = F, b_j = V, \forall j \ne i\}) $ \\
\vdots \\
$ c_{F,2} = ((x_{n-1,1}, x_{n,2}), \{(b_1, b_2)| b_1 = F, b_2 = V \lor b_1 = V, b_2 = F\}) \land ((x_{n-1,n}, x_{n,n-1}), \{(b_1, b_2)|b_1 = F, b_2 = V \lor b_1 = V, b_2 = F\}) $


\begin{equation}
\left( \begin{array}{ccccc}
c_{F,n} & c_{F,n+1} & \dots & c_{F,2*n-2} & \hfill \\
c_{F,n-1} & c_{F,n} & c_{F,n+1} & \vdots&  c_{F,2*n-2}\\
\vdots & c_{F,n-1} & c_{F,n} &c_{F,n+1} & \vdots \\
c_{F,2}& \vdots & c_{F,n-1}& c_{F,n} &c_{F,n+1} \\
\hfill & c_{F,2} & \dots& c_{F,n-1} & c_{F,n} \\
\end{array} \right)
\end{equation}

\textbf{Contraintes cavalier:}\\
$ c_{C,(i,j)} = ((x_{i,j}, x_{i+1,j+2}, x_{i+1,j-2}, x_{i-1,j+2}, x_{i-1,j-2}, x_{i+2,j+1}, x_{i+2,j-1}, x_{i-2,j+1}, x_{i-2,j-1}), \{(b_1, b_2, \ldots, b_9)|b_1 = C, b_2, \ldots b_9 = V\}) $


\section{Question 2}
L'ensemble des cases du jeux $V$ où $ \#V = n^2$ \\
\textbf {Variables de décision} $ X = \{x_{i,j} | \forall i,j (1 \leq i \leq n).(1 \leq j \leq n)\}$ , $n^2$ variables de décision \\
\textbf{Domaines : }
$ D = \{Vide, tour, fous, cavalier \} $ $D_i$ étant le domaine de la variable $x_{i,j}$\\


\textbf{Contraintes:}\\
%$ c = (\{x_{i,j} \forall i,j (1 \leq i \leq n).(1 \leq j \leq n)\}$ au moins un pions qui menace $x_{i,j} )$
%$ x_{i,j} = V \land $ \\
$ c_T = ((x_{i,j}, \forall i,j (1 \leq i \leq n).(1 \leq j \leq n)), \{b_{i,j}, \forall i,j (1 \leq i \leq n).(1 \leq j \leq n)|b_{i,j} = V, \forall l, (1 \leq l \leq n) \exists b_{l,j} = T \lor b_{i,j} = V, \forall k (1 \leq k \leq n), \exists b_{i,k} = T\})  $  \\
$ c_{couple,T} = ((x_{i,j}, x_{l,k}), \{(b_{i,j}, b_{l,k})|b_{i,j} = T, b_{l,k} = V, \rightarrow b_{i,m} \forall m (j < m < k), b_{i,m} = V \} \lor b_{i,j} = T, b_{l,k} = V, \rightarrow b_{m,j} \forall m (i < m < l), b_{m,j} = V) $\\

\textbf{Autre possibilité}\\
$ c_{couple,T_{colonnes}} = ((x_{i,j}, x_{l,j}), \{(b_{i,j}, b_{l,j})|b_{i,j} = T, b_{l,j} = V, \rightarrow b_{m,j} \forall m (i < m < l), b_{m,j} = V \}$\\
$ c_{couple,T_{lignes}} = ((x_{i,j}, x_{i,k}), \{(b_{i,j}, b_{i,k})|b_{i,j} = T, b_{i,k} = V, \rightarrow b_{i,m} \forall m (j < m < k), b_{i,m} = V \}$

\section{Question 3}

\section{Question Bonus}

\section{Question 4}

\section{Question 5}

\end{document}