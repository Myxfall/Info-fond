\documentclass[a4paper,10pt]{article}
\usepackage[utf8]{inputenc}
\usepackage[francais]{babel}
\usepackage{indentfirst}
\usepackage{listings}
\usepackage{graphicx}
\usepackage{blindtext}
\usepackage{enumitem}
\usepackage{hyperref}
\usepackage{amsmath}
\usepackage[top=2.5cm,bottom=2.5cm,left=2.5cm,right=2.5cm]{geometry}
\pagestyle{headings}
\title{Projet : INFO-F-302 Informatique Fondamentale.}
\author{George Rusu et Maximilien Romain}
\date{\today}
\begin{document}
\maketitle
\tableofcontents
\newpage
\section{Introduction}
Le premier objectif de ce projet est de modeliser divers problemes en problemes de satisfaction de contraintes (CSP). Le second objectif est d’implementer un programme resolvant ces problemes en utilisant ChocoSolver.

\textbf{link} :  http://www-master.ufr-info-p6.jussieu.fr/2005/IMG/pdf/rp3.pdf

\section{Question 1}
L'ensemble des cases du jeux $V$ où $ \#V = n^2$ \\
\textbf {Variables de décision} $ X = \{x_{i,j} | \forall i,j (1 \leq i \leq n).(1 \leq j \leq n)\}$ , $n^2$ variables de décision \\
\textbf{Domaines : } $ D = \{vide, fous, cavalier, tour\} $\\
\textbf{Contraintes : } Pour chacune des pièces de l'echequier, si une case est occupé par un pions, alors dans la porté de ce pion, donc les cases attaquables par ce pion, doivent être obligatoirement vide.  On parcours donc chaque case, et lorsque qu'une case est occupé, en fonction du pion qui occupe cette case, les contraintes changent.


\subsection{Contraintes tour}
	Pour chaque colonne de l'echequier, si une case de cette colonne est occupée par une tour, alors le reste des cases de cette colonne doivent être vide:
$$ c_{T_{col},j} = ((x_{1,j}, x_{2,j},\ldots, x_{n,j}), \{(b_1, b_2,\ldots, b_n) | b_i = T, b_j = V, \forall j \ne i\})$$
	Pour chaque ligne de l'echequier, si une case de cette ligne est occupée par une tour, alors le reste des cases de cette ligne doivent être vide:
$$ c_{T_{i,ligne}} = ((x_{i,1}, x_{i,2},\ldots, x_{i,n}), \{(b_1, b_2,\ldots, b_n) | b_i = T, b_j = V, \forall j \ne i\})$$

\subsection{Contraintes fou}
	Les contraintes du fou sont composées dans le même état d'esprit que les contraintes des tours, cepandant la portée d'un fou change, et en fonction de la digonale sur laquelle est placé le fou, la distance maximal que ce fou peut atteindre change.  En effet les plus grand diagonales se trouve en $(0,0)$ et $(0,n-1)$.
	Par contrainte, on considère toujours la diagonale opposée, et on réunie les deux diagonales par un \emph{and}.

$$ c_{F,2*n-2} = ((x_{1,n-1}, x_{2,n}), \{(b_1, b_2)| b_1 = F, b_2 = V \lor b_1 = V, b_2 = F\})\land$$ $$((x_{1,2}, x_{2,1}), \{(b_1, b_2)| b_1 = F, b_2 = V \lor b_1 = V, b_2 = F\}) $$ 
{\centering
  \vdots\par
}
$$ c_{F,n+1} = ((x_{1,2}, x_{2,3}, \ldots, x_{n-1,n}), \{(b_1, b_2, \ldots, b_{n-1})|b_i = F, b_j = V, \forall j \ne i\}) \land$$ $$((x_{1,n-1},x_{2,n-2}, \ldots, x_{n-1,1}), \{(b_1, b_2, \ldots, b_{n-1})|b_i = F, b_j = V, \forall j \ne i\}) $$
$$ c_{F,n} = ((x_{1,1}, x_{2,2},\ldots, x_{n,n}), \{(b_1, b_2, \ldots, b_n)|b_i = F, b_j = V, \forall j \ne i\}) \land$$ $$((x_{1,n}, x_{2,n-1}, \ldots, x_{n, 1}), \{(b_1, b_2, \ldots, b_n)|b_i = F, b_j = V, \forall j \ne i\})$$
$$ c_{F,n-1} = ((x_{2,1}, x_{3,2},\ldots, x_{n, n-1}), \{(b_1, b_2, \ldots, b_{n-1})|b_i = F, b_j = V, \forall j \ne i\}) \land$$ $$((x_{2,n}, x_{3,n-1}, \ldots, x_{n,2}), \{(b_1, b_2, \ldots, b_{n-1})|b_i = F, b_j = V, \forall j \ne i\}) $$
{\centering
  \vdots\par
}
$$ c_{F,2} = ((x_{n-1,1}, x_{n,2}), \{(b_1, b_2)| b_1 = F, b_2 = V \lor b_1 = V, b_2 = F\}) \land$$ $$((x_{n-1,n}, x_{n,n-1}), \{(b_1, b_2)|b_1 = F, b_2 = V \lor b_1 = V, b_2 = F\}) $$

	Les diagonales de gauche à droite sont représenté par la matrice ci-dessous.
\begin{equation}
\left( \begin{array}{ccccc}
c_{F,n} & c_{F,n+1} & \dots & c_{F,2*n-2} & \hfill \\
c_{F,n-1} & c_{F,n} & c_{F,n+1} & \vdots&  c_{F,2*n-2}\\
\vdots & c_{F,n-1} & c_{F,n} &c_{F,n+1} & \vdots \\
c_{F,2}& \vdots & c_{F,n-1}& c_{F,n} &c_{F,n+1} \\
\hfill & c_{F,2} & \dots& c_{F,n-1} & c_{F,n} \\
\end{array} \right)
\end{equation}

\subsection{Contraintes cavalier:}
	Les contraintes des cavaliers sont plus simple. En effet si on rencontre une case occupée par un cavalier, alors il n'y a que 8 contraintes pour les 8 cases que le cavalier peut atteindre.  Si le cavalier est en $(i,j)$, alors il ne peut qu'attaquer les cases où $i+e,j+e, e \in [+1,-1,+2,-2]$
$$ c_{C,(i,j)} = ((x_{i,j}, x_{i+1,j+2}, x_{i+1,j-2}, x_{i-1,j+2}, x_{i-1,j-2}, x_{i+2,j+1}, x_{i+2,j-1}, x_{i-2,j+1}, x_{i-2,j-1}),$$ $$\{(b_1, b_2, \ldots, b_9)|b_1 = C, b_2, \ldots b_9 = V\}) $$


\section{Question 2}
L'ensemble des cases du jeux $V$ où $ \#V = n^2$ \\
\textbf {Variables de décision} $ X = \{x_{i,j} | \forall i,j (1 \leq i \leq n).(1 \leq j \leq n)\}$ , $n^2$ variables de décision \\
\textbf{Domaines : }
$ D = \{Vide, tour, fous, cavalier \} $\\
\textbf{Contraintes : } Pour chacune des pièces de l'echequier, si une case est vide, alors il existe au moins un pion qui menace cette case vide

\subsection{Contraintes tours}
	Pour chaque case de l'echequier qui n'est pas occupé par un pion, alors il existe au moins une tour dans la ligne ou la colonne de la case non occupée.
$$ c_T = ((x_{i,j}, \forall i,j (1 \leq i \leq n).(1 \leq j \leq n)),$$ $$\{b_{i,j}, \forall i,j (1 \leq i \leq n).(1 \leq j \leq n)|b_{i,j} = V, \forall l, (1 \leq l \leq n) \exists b_{l,j} = T \lor b_{i,j} = V, \forall k (1 \leq k \leq n), \exists b_{i,k} = T\})  $$
	Nous devons aussi indiqué qu'il n'y pas d'autre pion entre la case vide et la tour.  On indique donc que les cases entre la case vide et la case occupé par une tour, doivent non occupées. Nous avons donc une telle contrainte pour la ligne et la colonne de la case vide menacé par une tour.
%$$ c_{couple,T} = ((x_{i,j}, x_{l,k}), \{(b_{i,j}, b_{l,k})|b_{i,j} = T, b_{l,k} = V, \rightarrow b_{i,m} \forall m (j < m < k), b_{i,m} = V \} \lor$$ $$b_{i,j} = T, b_{l,k} = V, \rightarrow b_{m,j} \forall m (i < m < l), b_{m,j} = V) $$
$$ c_{couple,T_{colonnes}} = ((x_{i,j}, x_{l,j}), \{(b_{i,j}, b_{l,j})|b_{i,j} = T, b_{l,j} = V, \rightarrow b_{m,j} \forall m (i < m < l), b_{m,j} = V \})$$
$$ c_{couple,T_{lignes}} = ((x_{i,j}, x_{i,k}), \{(b_{i,j}, b_{i,k})|b_{i,j} = T, b_{i,k} = V, \rightarrow b_{i,m} \forall m (j < m < k), b_{i,m} = V \})$$

\subsection{Contraintes fous}
	Pour chaque case de l'echequier qui n'est pas occupé par un pion, alors il existe au moins un fou dans les deux diagonales de la case vide.
$$ c_F = ((x_{i,j}, \forall i,j (1 \leq i \leq n).(1 \leq j \leq n)),$$ $$\{b_{i,j},\forall i,j (1 \leq i \leq n).(1 \leq j \leq n)|b_{i,j} = V, \forall l,k (1 \leq l \leq n).(1 \leq k \leq n) \exists b_{l,k} = F \lor$$ $$b_{i,j} = V, \forall l,k (1 \leq l \leq n).(n \leq k \leq 1), \exists b_{l,k} = F\})  $$
	De nouveau nous devons nous assurer qu'il n'y a pas d'autre pions qui gène le fou entre la case vide et ce fou.
$$ c_{couple,F_{diag1}} = ((x_{i,j}, x_{l,k}), \{(b_{i,j}, b_{l,k})|b_{i,j} = T, b_{l,k} = V, \rightarrow b_{x,y} \forall x,y (i < x < l).(j < y < k), b_{x,y} = V \})$$
$$ c_{couple,F_{diag2}} = ((x_{i,j}, x_{l,k}), \{(b_{i,j}, b_{l,k})|b_{i,j} = T, b_{l,k} = V, \rightarrow b_{x,y} \forall x,y (i < x < l).(k < y < j), b_{x,y} = V \})$$

\subsection{Contraintes cavaliers:}
		Pour chaque case de l'echequier qui n'est pas occupé par un pion, alors il existe au moins un cavalier qui menace la case vide. Pour cela il suffit de verifier qu'il existe une case dans les 8 cases correspondant à l'ensemble des mouvements du cavalier ($\{+1,-1,+2,-2\}$), soit occupé par un cavalier.
$$ c_{C,(i,j)} = ((x_{i,j}, x_{i+1,j+2}, x_{i+1,j-2}, x_{i-1,j+2}, x_{i-1,j-2}, x_{i+2,j+1}, x_{i+2,j-1}, x_{i-2,j+1}, x_{i-2,j-1}),$$ $$\{(b_1, b_2, \ldots, b_9)|b_1 = V, \forall i,j \in [+1,+2,-1,-2], \exists b_{i,j} = C\}) $$

\subsection{Contrainte finale:}
	Finalement il n'y a qu'une seule grande contrainte à appliquer à notre problème. Etant donné qu'il suffit qu'un seule pion menace une case vide, nous pouvons relier nos contrainte par des conditions \emph{or}. 
$$ C = (C_T \land c_{couple,T_{colonnes}} \land c_{couple,T_{lignes}}) \lor (c_F \land c_{couple,F_{diag1}} \land c_{couple,F_{diag2}}) \lor c_{C,(i,j)}$$

\section{Question 2 avec ensemble}
L'ensemble des cases du jeux $V$ où $ \#V = n^2$ \\
\textbf {Variables de décision} $ X = \{x_{i,j} | \forall i,j (1 \leq i \leq n).(1 \leq j \leq n)\}$ , $n^2$ variables de décision \\
\textbf{Domaines : }
$ D = \{Vide, tour, fous, cavalier \} $\\
\textbf{Contraintes : } Pour chacune des pièces de l'echequier, si une case est vide, alors il existe au moins un pion qui menace cette case vide.  Le problème ici, est que nous devons trouver un moyen d'appliquer un \emph{OR} de contraintes, ce qui n'est pas possible.  Nous allons donc définir divers ensemble, et ensuite écrire une grande contrainte appliquant des \emph{Unions} de ces ensembles.

	\subsection{Ensemble Tours}
		Pour chaque case de l'echequier qui n'est pas occupé par un pion, alors il existe au moins une tour dans la ligne ou la colonne de la case non occupée.
		$$ e_T = \{(x_{i,j}, \forall i,j (1 \leq i \leq n).(1 \leq j \leq n)),$$ $$\{b_{i,j}, \forall i,j (1 \leq i \leq n).(1 \leq j \leq n)|b_{i,j} = V, \forall l, (1 \leq l \leq n) \exists b_{l,j} = T \lor b_{i,j} = V, \forall k (1 \leq k \leq n), \exists b_{i,k} = T\}\}  $$
			Nous devons aussi indiqué qu'il n'y pas d'autre pion entre la case vide et la tour.  On indique donc que les cases entre la case vide et la case occupé par une tour, doivent non occupées. Nous avons donc une telle contrainte pour la ligne et la colonne de la case vide menacé par une tour.
		%$$ c_{couple,T} = ((x_{i,j}, x_{l,k}), \{(b_{i,j}, b_{l,k})|b_{i,j} = T, b_{l,k} = V, \rightarrow b_{i,m} \forall m (j < m < k), b_{i,m} = V \} \lor$$ $$b_{i,j} = T, b_{l,k} = V, \rightarrow b_{m,j} \forall m (i < m < l), b_{m,j} = V) $$
		$$ e_{couple,T_{colonnes}} = \{(x_{i,j}, x_{l,j}), \{(b_{i,j}, b_{l,j})|(b_{i,j} = T, b_{l,j} = V) \land ( \forall m (i < m < l), b_{m,j} = V) \}\}$$
		$$ e_{couple,T_{lignes}} = \{(x_{i,j}, x_{i,k}), \{(b_{i,j}, b_{i,k})|(b_{i,j} = T, b_{i,k} = V) \land ( \forall m (j < m < k), b_{i,m} = V) \}\}$$

	\subsection{Ensemble fous}
		Pour chaque case de l'echequier qui n'est pas occupé par un pion, alors il existe au moins un fou dans les deux diagonales de la case vide.
		$$ e_F = \{(x_{i,j}, \forall i,j (1 \leq i \leq n).(1 \leq j \leq n)),$$ $$\{b_{i,j},\forall i,j (1 \leq i \leq n).(1 \leq j \leq n)|b_{i,j} = V, \forall l,k (1 \leq l \leq n).(1 \leq k \leq n) \exists b_{l,k} = F \lor$$ $$b_{i,j} = V, \forall l,k (1 \leq l \leq n).(n \leq k \leq 1), \exists b_{l,k} = F\}\}  $$
			De nouveau nous devons nous assurer qu'il n'y a pas d'autre pions qui gène le fou entre la case vide et ce fou.
		$$ e_{couple,F_{diag1}} = \{(x_{i,j}, x_{l,k}), \{(b_{i,j}, b_{l,k})|(b_{i,j} = T, b_{l,k} = V) \land ( \forall x,y (i < x < l).(j < y < k), b_{x,y} = V) \}\}$$
		$$ e_{couple,F_{diag2}} = \{(x_{i,j}, x_{l,k}), \{(b_{i,j}, b_{l,k})|(b_{i,j} = T, b_{l,k} = V) \land ( \forall x,y (i < x < l).(k < y < j), b_{x,y} = V) \}\}$$

	\subsection{Ensemble cavaliers:}
			Pour chaque case de l'echequier qui n'est pas occupé par un pion, alors il existe au moins un cavalier qui menace la case vide. Pour cela il suffit de verifier qu'il existe une case dans les 8 cases correspondant à l'ensemble des mouvements du cavalier ($\{+1,-1,+2,-2\}$), soit occupé par un cavalier.
	$$ e_{C,(i,j)} = \{(x_{i,j}, x_{i+1,j+2}, x_{i+1,j-2}, x_{i-1,j+2}, x_{i-1,j-2}, x_{i+2,j+1}, x_{i+2,j-1}, x_{i-2,j+1}, x_{i-2,j-1}),$$ $$\{(b_1, b_2, \ldots, b_9)|b_1 = V, \forall i,j \in \{+1,+2,-1,-2\}, \exists b_{i,j} = C\}\} $$

	\subsection{Contrainte finale:}
		Finalement il n'y a qu'une seule grande contrainte à appliquer à notre problème. Etant donné qu'il suffit qu'un seule pion menace une case vide, nous pouvons relier nos ensemble par des \emph{unions} $\cup$ et des \emph{intersections} de contraintes $\cap$. 
	$$ C = (e_T \cap e_{couple,T_{colonnes}} \cap e_{couple,T_{lignes}}) \cup (e_F \cap e_{couple,F_{diag1}} \cap e_{couple,F_{diag2}}) \cup e_{C,(i,j)}$$

\section{Question 3}
	\subsection{Fichier main.java}

\section{Question Bonus}

\section{Question 4}

\section{Question 5}

\end{document}